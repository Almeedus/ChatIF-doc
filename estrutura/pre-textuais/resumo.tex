% RESUMO--------------------------------------------------------------------------------

\begin{resumo}[RESUMO]
\begin{SingleSpacing}

% Auto citação completa--------------------------------------------------------
%\imprimirautorcitacao. \imprimirtitulo. \imprimirdata. \pageref {LastPage} f. \imprimirprojeto\ – %\imprimirprograma, \imprimirinstituicao. \imprimirlocal, \imprimirdata.\\
%---------------------------------------------------------------------------------------

%O Resumo é um elemento obrigatório em tese, dissertação, monografia e TCC, constituído de uma seqüência de frases concisas e objetivas, fornecendo uma visão rápida e clara do conteúdo do estudo. O texto deverá conter no máximo 500 palavras e ser antecedido
%pela referência do estudo. Também, não deve conter citações. O resumo deve ser redigido em parágrafo único, espaçamento simples e seguido das palavras representativas do conteúdo do estudo, isto é, palavras-chave, em número de três a cinco, separadas entre si por ponto e finalizadas também por ponto. Usar o verbo na terceira pessoa do singular, com linguagem impessoal, bem como fazer uso, preferencialmente, da voz ativa. Texto contendo um único parágrafo.\\

O presente trabalho teve como objetivo desenvolver um sistema de chatbot com 
inteligência artificial voltado ao atendimento de dúvidas sobre o processo 
seletivo do Instituto Federal de São Paulo – Campus Itapetininga. A proposta 
busca facilitar a obtenção de informações contidas nos editais, reduzindo o 
tempo de busca e promovendo maior clareza na comunicação com os candidatos. A 
metodologia adotada envolveu o uso do modelo GPT-4o, treinado com o edital 
oficial, integrado via LangChain e consumido por uma API desenvolvida em Python 
com FastAPI. Os dados foram persistidos em um banco Redis, permitindo a 
posterior exportação e reuso para novos treinamentos. A interface foi 
implementada em Vue.js visando acessibilidade e simplicidade. Para validação 
do sistema, foi realizada uma pesquisa com dois grupos de usuários, que revelou 
maior eficiência do chatbot em perguntas complexas em comparação à leitura 
direta do edital. O sistema demonstrou potencial para melhorar o atendimento 
institucional e otimizar o tempo dos candidatos.

\textbf{Palavras-chave}: Inteligência Artificial. Chatbot. LLM. Aplicação WEB.

\end{SingleSpacing}
\end{resumo}

% OBSERVAÇÕES---------------------------------------------------------------------------
% Altere o texto inserindo o Resumo do seu trabalho.
% Escolha de 3 a 5 palavras ou termos que descrevam bem o seu trabalho 

