% ABSTRACT--------------------------------------------------------------------------------

\begin{resumo}[ABSTRACT]
\begin{SingleSpacing}

% Auto-Citação completa em Ingles--------------------------------------------------------
%\imprimirautorcitacao. \imprimirtitleabstract. \imprimirdata. \pageref {LastPage} f. \imprimirprojeto\ – %\imprimirprograma, \imprimirinstituicao. \imprimirlocal, \imprimirdata.\\
%---------------------------------------------------------------------------------------

This work aimed to develop an artificial intelligence-powered chatbot system to 
assist with answering questions regarding the admission process of the Instituto 
Federal de São Paulo – Itapetininga Campus. The main goal was to facilitate 
access to the information presented in official announcements, reducing the 
search time and improving clarity for prospective students. The methodology 
involved using the GPT-4o model, fine-tuned with the official admission notice, 
integrated via LangChain and accessed through a Python-based API built with 
FastAPI. Data was stored in a Redis database, allowing for export and reuse in 
future training sessions. The frontend was developed using Vue.js to ensure 
accessibility and ease of use. To validate the system, a user study was 
conducted comparing traditional PDF consultation with the chatbot. The results 
indicated that the chatbot was more effective for complex queries, 
demonstrating its potential to enhance institutional support and optimize 
candidate experience.

\textbf{Keywords}: Artificial intelligence. Chatbot. LLM. WEB application.
\end{SingleSpacing}
\end{resumo}

% OBSERVAÇÕES---------------------------------------------------------------------------
% Altere o texto inserindo o Abstract do seu trabalho.
% Escolha de 3 a 5 palavras ou termos que descrevam bem o seu trabalho 
