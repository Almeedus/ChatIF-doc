% ABSTRACT--------------------------------------------------------------------------------

\begin{resumo}[ABSTRACT]
\begin{SingleSpacing}

% Auto-Citação completa em Ingles--------------------------------------------------------
%\imprimirautorcitacao. \imprimirtitleabstract. \imprimirdata. \pageref {LastPage} f. \imprimirprojeto\ – %\imprimirprograma, \imprimirinstituicao. \imprimirlocal, \imprimirdata.\\
%---------------------------------------------------------------------------------------

This work aimed to develop an artificial intelligence chatbot system focused on 
answering questions about the selection process for the Federal Institute of 
São Paulo – Itapetininga Campus. The proposal seeks to facilitate the retrieval 
of information contained in the notices, reducing search time and promoting 
greater clarity in communication with candidates. The methodology adopted 
involved the use of the GPT-40 model, trained with the official notice, 
integrated via LangChain and consumed by an API developed in Python with 
FastAPI. The data was persisted in a Redis database, allowing for later export 
and reuse for new training. The interface was implemented in Vue.js aiming for 
accessibility and simplicity. For system validation, tests were carried out 
during development to verify its efficiency in searching for information from 
the notice.

\textbf{Keywords}: Artificial intelligence. Chatbot. LLM. WEB application.
\end{SingleSpacing}
\end{resumo}

% OBSERVAÇÕES---------------------------------------------------------------------------
% Altere o texto inserindo o Abstract do seu trabalho.
% Escolha de 3 a 5 palavras ou termos que descrevam bem o seu trabalho 
