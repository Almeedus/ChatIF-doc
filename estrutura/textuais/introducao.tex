% INTRODUÇÃO-------------------------------------------------------------------

\chapter{INTRODUÇÃO}
\label{chap:introducao}

\textual
\setcounter{page}{11}

Buscou-se com o desenvolvimento deste trabalho disponibilizar um sistema de 
chatbot com inteligência artificial para responder questões sobre o processo 
seletivo no Instituto Federal de São Paulo para o Campus de Itapetininga. 
A ideia teve como intuito reduzir pela metade o tempo de procura pelas 
informações presentes nos editais.

Por meio de uma análise nos editais oficiais acessados na página do IFSP, 
verificou-se que o candidato dedicava um tempo significativo para buscar 
informações relacionadas a diversos tópicos envolvidos no processo seletivo, 
como pagamento, inscrição e matrícula. Esse tempo poderia ser melhor empregado 
na realização das etapas necessárias ao ingresso na instituição.

Observou-se que a linguagem formal aplicada na redação dos editais dificultava 
a compreensão por parte de alguns candidatos. Dessa forma, um sistema que 
fornecesse respostas de maneira mais clara, objetiva e simples reduziria o 
tempo necessário para compreender as informações contidas nos documentos oficiais.

Esse cenário ocorre em decorrência do analfabetismo funcional, condição na qual 
o indivíduo é capaz de realizar leituras curtas ou interpretar frases isoladas, 
mas apresenta dificuldades na compreensão de textos simples ou na resolução de 
operações mais complexas. De acordo com pesquisa apresentada por 
\cite{educamais_analfabetismo}, houve um crescimento desse índice de 27\% em 2018 
para 29\% em 2024. Diante desse aumento no número de analfabetos funcionais, o 
desenvolvimento do sistema proposto surge como uma alternativa para auxiliar na 
leitura e compreensão de conteúdos mais complexos, promovendo maior acessibilidade 
à informação.

A escolha do tema ocorreu devido à popularização da inteligência artificial 
em diversas áreas, sendo uma delas a acadêmica como destaca a Fundação 
Coordenação de Aperfeiçoamento de Pessoal de Nível Superior 
\citeonline{capes_ia_estudos} ao informar que o Brasil está entre as 20 nações 
que mais publicam trabalhos voltados ao tema entre os anos de 2019 e 2023, o 
que equivalem a 6,3 mil estudos que tem impulsionado o surgimento constante de 
novas ferramentas e tecnologias capazes de otimizar processos. A utilização de 
um sistema de inteligência artificial proporcionaria uma forma de consulta mais 
intuitiva, por meio de um layout baseado em perguntas e respostas. 
As informações seriam apresentadas de forma precisa, uma vez que o modelo 
empregado foi treinado com questões voltadas ao próprio edital, garantindo 
maior eficiência na busca por informações.

Além da otimização do tempo do candidato, também se buscou reduzir a demanda 
dos colaboradores nos canais de comunicação, como e-mail, telefone e WhatsApp, 
no esclarecimento de dúvidas sobre o processo seletivo. Dessa forma, a 
necessidade de contato direto com os atendentes ocorreria apenas nos casos em 
que o sistema não conseguisse fornecer a resposta desejada, direcionando o 
usuário a um suporte via e-mail.