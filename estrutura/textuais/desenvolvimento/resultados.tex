% RESULTADOS-------------------------------------------------------------------

\chapter{DESENVOLVIMENTO E RESULTADOS}

O presente trabalho foi considerado uma pesquisa de natureza aplicada, pois buscou, através do desenvolvimento de um sistema utilizando métodos e tecnologias atuais, solucionar o problema apontado neste documento.

A pesquisa abordada foi descritiva e exploratória, pois utilizou metodologias de observação para definir o tempo de pesquisa da população diante do sistema utilizado e da pesquisa tradicional no PDF, para assim metrificar a utilização de ambas as formas de pesquisa e verificar se o usuário diante do sistema desenvolvido apresentou um tempo menor para realizar a busca.

No que se refere à abordagem, este projeto se enquadrou num contexto onde duas metodologias foram empregadas: abordagem qualitativa, por meio da observação da população, e abordagem quantitativa, para delimitar com dados o impacto do sistema desenvolvido diante da observação realizada.

A principal forma de coleta de dados ocorreu no Instituto Federal de São Paulo, Campus de Itapetininga, onde foram investigadas duas populações:

\begin{itemize}
    \item Grupo A: 5 indivíduos que buscaram informações sobre o processo seletivo utilizando o sistema desenvolvido.
    \item Grupo B: 5 indivíduos que realizaram consultas no edital do processo seletivo para encontrar as informações.
\end{itemize}

As perguntas fornecidas na pesquisa para as populações foram as seguintes: 

\begin{enumerate}
    \item Qual é o valor para realizar a inscrição no vestibular?
    \item Quais documentos são necessários para fazer a matricula?
    \item Até que dia posso me inscrever para realizar a prova?
    \item Quando sairá a data a respeito do local de prova do Campus de Itapetininga?
    \item Caso eu passe na prova, quando posso fazer a matricula?
\end{enumerate}

Diante da observação das populações foi verificado que o sistema se faz uma ótima opção quando a busca é por informações mais complexas como é o caso das perguntas 2 e 4, entretanto o sistema não é recomendado para perguntas que são mais visíveis no edital. Dessa forma o sistema útil em situações específicas, como pode ser analisado ao obter a média aritmética simples do tempo levado na pesquisa em cada.

\begin{tabular}{ |p{2cm}|p{2cm}|p{2cm}| }
\hline
\multicolumn{3}{|c|}{Média Aritmética Simples} \\
\hline
 & Grupo A & Grupo B \\
\hline
Pergunta 1 & 00:14.2 & 00:09.4 \\
Pergunta 2 & 00:17.6 & 01:56.2 \\
Pergunta 3 & 00:13.6 & 00:28.2 \\
Pergunta 4 & 00:10.8 & 01:00.4 \\
Pergunta 5 & 00:16.0 & 00:12.6 \\
\hline
\end{tabular}

A pesquisa foi conduzida por meio da cronometragem do tempo gasto para encontrar as respostas às perguntas fornecidas. Ao final, a avaliação dos resultados ocorreu de forma quantitativa, delimitando o tempo gasto em cada pergunta e no total, e qualitativa, ao identificar as dificuldades encontradas durante o processo de busca por informações para pensar em melhorias.