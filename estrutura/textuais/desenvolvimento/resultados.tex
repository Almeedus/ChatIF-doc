% RESULTADOS-------------------------------------------------------------------

\chapter{DESENVOLVIMENTO E RESULTADOS}

O presente trabalho caracteriza-se como uma pesquisa de natureza aplicada, uma vez que buscou, por meio do desenvolvimento de um sistema utilizando métodos e tecnologias atuais, propor uma solução prática para o problema identificado ao longo deste estudo.

O sistema foi estruturado em uma arquitetura composta por três principais camadas: uma interface web desenvolvida em Vue.js, responsável pela interação com o usuário; uma API em Python utilizando o framework FastAPI, responsável pelo processamento das requisições e pela comunicação com o modelo de linguagem; e um banco de dados Redis, utilizado para armazenar as interações realizadas. Essa divisão possibilitou uma comunicação eficiente entre os componentes, maior organização do código e flexibilidade para futuras melhorias.

Durante a construção do código e a configuração do modelo, foram encontrados alguns desafios, especialmente relacionados ao consumo de APIs externas, que exigem o envio de tokens e, consequentemente, geram custos por utilização. O desenvolvimento das chains (cadeias de execução) mostrou-se um processo interessante e didático, permitindo estruturar uma lógica simples e eficiente para o fluxo do sistema.

Um dos principais diferenciais deste projeto foi a utilização do processo de Fine Tuning disponibilizado pela OpenAI, que permitiu o treinamento de um modelo ajustado às necessidades do contexto escolar. Esse processo exigiu a preparação de um conjunto de dados no formato JSONL, contendo exemplos de perguntas e respostas que orientaram o comportamento do modelo. Além disso, foi necessário configurar parâmetros de treinamento, como taxa de aprendizado, número de épocas e tamanho do lote, de forma a otimizar os resultados obtidos.

No uso de um banco de dados não relacional, um dos desafios mais relevantes foi a formatação adequada dos arquivos JSON. Durante a implementação da lógica em Python responsável por persistir as informações, observou-se que a codificação dos caracteres não era mantida corretamente, resultando em inconsistências, principalmente com caracteres especiais. Assim, as informações retornavam com símbolos incorretos. Com o apoio de fóruns e da comunidade de desenvolvedores, foi possível identificar a causa do problema e implementar uma solução eficiente utilizando codificação no padrão UTF-8, garantindo a integridade dos dados armazenados e a correta leitura das informações posteriormente.

Após as configurações e ajustes, o sistema demonstrou respostas mais coerentes e contextualizadas às perguntas dos usuários, especialmente nas consultas relacionadas ao ambiente escolar, documentos e procedimentos institucionais. O desempenho geral manteve-se satisfatório, apresentando baixo tempo de resposta mesmo em situações de múltiplas requisições simultâneas.

O desenvolvimento do sistema proporcionou uma compreensão aprofundada sobre o funcionamento de APIs, bancos de dados não relacionais e integração com modelos de linguagem. Além disso, reforçou a importância do tratamento adequado de dados, da organização modular do código e da adoção de boas práticas de codificação para garantir a confiabilidade, escalabilidade e manutenção da aplicação.