% METODOLOGIA------------------------------------------------------------------

\chapter{METODOLOGIA}
\label{chap:metodologia}

O trabalho proposto tem o objetivo geral sendo o \textbf{desenvolver} um sistema 
web que auxilia os candidatos no processo seletivo do IFSP cujo o 
\textit{backend} consome um \textbf{modelo de linguagem} (LLM) 
\textbf{treinado via \textit{FineTuning}} pela plataforma da OpenAI, utilizando
como conexão uma API construída em Python integrada a um banco de dados não
relacional que armazena as interações.

Um dos objetivos específicos são o \textbf{desenvolvimento} de uma interface 
\textit{web} construída em VueJS que permite a interação dos candidatos com o
sistema de forma acessível e intuitiva. Como segundo objetivo específico é a 
\textbf{avaliação do desempenho do sistema} por meio de testes durante o seu 
desenvolvimento, verificando assim a eficácia no auxílio aos candidatos durante
a utilização no processo seletivo.

Durante a construção do trabalho houve a pesquisa em fóruns de comunidades,
artigos acadêmicos e também em documentações oficiais das linguagens e 
\textit{frameworks}.

Devido ao suporte fornecido pelos sistemas operacionais Linux e Windows para o 
banco de dados em questão, a \textit{stack} de tecnologias tem uma pequena 
distinção, ao utilizar do Windows é necessário o recurso WSL 
(\textit{Windows Subsystem for Linux} ou aportuguesando Subsistema Windows para 
Linux) para executar um ambiente Linux sem que seja necessário a utilização de 
uma máquina virtual ou uma dupla inicialização para se usar o banco de dados 
Redis. O \textit{backend} fica composto pelo banco de dados instalado dentro do 
WSL que se comunica diretamente com a API desenvolvida em python com o 
\textit{LangChain}, que por sua vez faz a comunicação com o \textit{frontend} 
que é a interface construída em VueJS como aponta a 
\autoref{fig:diagrama-stack-windows}. 

%Stack do Windows
\begin{figure}[!htb]
    \centering
    \caption{\textit{Stack} de Tecnologias para \textit{Windows}}
    \includegraphics[width=0.9\textwidth]{./dados/figuras/diagrama-stack-windows}
    \fonte{Elaborado pelo Autor}
    \label{fig:diagrama-stack-windows}
\end{figure}

Pensando numa arquitetura em Linux, o \textit{backend} é composto pelo banco de 
dados Redis que se comunica com a API \textit{Python} com \textit{LangChain}, e 
seu consumo é feito pelo \textit{frontend} desenvolvido em VueJS, descrito 
assim pela \autoref{fig:diagrama-stack-linux}. Dessa forma ao utilizar o Linux 
não se faz necessário a utilização do WSL, uma vez que o Linux da suporte ao 
banco de dados utilizado no projeto.

%Stack do Linux
\begin{figure}[!htb]
    \centering
    \caption{Stack de Tecnologias \textit{Linux}}
    \includegraphics[width=0.9\textwidth]{./dados/figuras/diagrama-stack-linux}
    \fonte{Elaborado pelo Autor}
    \label{fig:diagrama-stack-linux}
\end{figure}