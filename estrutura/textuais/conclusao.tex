% CONCLUSÃO--------------------------------------------------------------------

\chapter{CONCLUSÃO}
\label{chap:conclusao}

O presente trabalho teve como objetivo desenvolver um sistema de chatbot utilizando inteligência artificial para auxiliar candidatos no processo seletivo do IFSP – Campus Itapetininga, tornando se uma alternativa mais acessível e rápida para a consulta de informações presentes nos editais oficiais.

Por meio da aplicação de tecnologias atuais como LangChain, FastAPI, GPT-4o e banco de dados Redis, foi possível construir uma solução funcional, testada com usuários reais. Os resultados demonstraram que, para perguntas de maior complexidade e interpretação, o chatbot apresentou desempenho melhor do que diante da leitura direta dos editais, reduzindo significativamente o tempo de obtenção das respostas.

O desenvolvimento deste projeto também reforça a importância do uso de modelos de linguagem de grande escala e técnicas de fine-tuning no contexto educacional, ampliando as possibilidades de aplicação da IA em ambientes acadêmicos.

Conclui-se, portanto, que o sistema atendeu parcialmente os objetivos propostos, demonstrando viabilidade técnica e impacto positivo na experiência do usuário. Ainda assim, há espaço para melhorias, como refinamentos na interface e mecanismos de tolerância a erros de digitação, que podem ser abordados em trabalhos futuros.

\section{TRABALHOS FUTUROS}
\label{sec:trabalhosFuturos}

Para uma melhor experiência do usuário é interessante aplicar a heurística de Nielsen que fornecem uma maneira de desenvolver uma interface considerada boa e dessa forma uma melhor experiência para quem a usa, como destaca \citeonline{nielsen_heuristicas_2025}. A primeira heurística se refere a visibilidade do status do sistema no momento em que está sendo processado uma resposta, ou seja, desenvolver um mensagem "criando sua resposta" ou até mesmo um estado de carregamento fornece para o usuário um retorno de forma visual.

Outro ponto importante em sua melhoria é o processamento do texto se houver algum erro em sua escrita, durante os testes uma pessoa da população estudada apresentou durante a pesquisa alguns erros e nesse caso em específico, o sistema não conseguiu entender a pergunta feita.

