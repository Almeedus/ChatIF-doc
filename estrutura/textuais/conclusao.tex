% CONCLUSÃO--------------------------------------------------------------------

\chapter{CONCLUSÃO}
\label{chap:conclusao}

O presente trabalho teve como objetivo desenvolver um sistema de chatbot utilizando inteligência artificial para auxiliar candidatos no processo seletivo do IFSP – Campus Itapetininga, tornando se uma alternativa mais acessível e rápida para a consulta de informações presentes nos editais oficiais.

Por meio da aplicação de tecnologias atuais como LangChain, FastAPI, GPT-4o e banco de dados Redis, foi possível construir uma solução funcional, testada com usuários reais. Os resultados demonstraram que, para perguntas de maior complexidade e interpretação, o chatbot apresentou desempenho melhor do que diante da leitura direta dos editais, reduzindo significativamente o tempo de obtenção das respostas.

O desenvolvimento deste projeto também reforça a importância do uso de modelos de linguagem de grande escala e técnicas de fine-tuning no contexto educacional, ampliando as possibilidades de aplicação da IA em ambientes acadêmicos.

Conclui-se, portanto, que o sistema atendeu parcialmente os objetivos propostos, demonstrando viabilidade técnica e impacto positivo na experiência do usuário. Ainda assim, há espaço para melhorias, como refinamentos na interface e mecanismos de tolerância a erros de digitação, que podem ser abordados em trabalhos futuros.

\section{TRABALHOS FUTUROS}
\label{sec:trabalhosFuturos}

O presente trabalho poderá originar dois artigos científicos 
complementares. O primeiro poderá abordar, de forma detalhada, todo o processo de 
criação, configuração e ajustes realizados durante o desenvolvimento do modelo de 
linguagem LLM. Já o segundo terá como foco a validação dos resultados obtidos, por 
meio de uma pesquisa aplicada com usuários, a fim de avaliar a efetividade e 
usabilidade do sistema proposto. Para a realização dessa etapa, será necessária 
a submissão de um projeto de pesquisa ao Comitê de Ética em Pesquisa (CEP), 
considerando o que haverá a participação de seres humanos. Parte da documentação 
exigida para esse processo já se encontra parcialmente preparada, facilitando 
assim o envio por meio da Plataforma Brasil.

Além disso, como proposta de aprimoramento no sistema, pretende-se melhorar o 
processamento textual, tornando-o capaz de reconhecer e tratar automaticamente 
erros de digitação ou de escrita. Essa melhoria visa permitir 
que o modelo identifique falhas na entrada do usuário e solicite a correção ou 
reescrita da mensagem, aumentando a eficiência e acessibilidade da interação 
com o chatbot.

Outro avanço planejado é a implementação de modelos locais e offline, utilizando 
a ferramenta Ollama, que possibilita a criação e execução de instâncias de 
modelos de linguagem diretamente no ambiente local, sem a necessidade de conexão 
constante com servidores externos. Entre os modelos disponíveis estão o 
LLaMA 3.1, GPT-OSS, DeepSeek-R1, entre outros, cada um projetado para diferentes 
finalidades e contextos de aplicação. Essa abordagem busca reduzir custos de uso 
de APIs, aumentar a privacidade dos dados e permitir testes comparativos de 
desempenho entre modelos locais e em nuvem.